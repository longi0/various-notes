\providecommand{\main}{../../}
\documentclass[\main/main.tex]{subfiles}
\begin{document}

\subsection{Cosa si intende per clustering? In quali famiglie vengono divisi? [3]}
Per clustering si intende raggruppamento degli input per similitudini. Ci sono vari famiglie di clustering: algoritmi gerarchici, algoritmi statistici, algoritmi partizionali hard-clustering e soft-clustering.

\subsection{Che relazione c'è tra clustering e classificazione}
Mentre il clustering cerca di raggruppare in cluster a partire dalle feature di un gruppo di oggetti, la classificazione si occupa, date le classi, di assegnare ogni elemento di un gruppo di oggetti ad una classe.

\subsection{Quali sono le criticità? [3]}
Il clustering non è un problema ben posto, perché ci sono diversi gradi di libertà da fissare su come effettuare un clustering, per esempio variando algoritmo, calcolo di feature, misura di similarità il risultato può essere molto diverso. 

\end{document}
